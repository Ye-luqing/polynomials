\documentclass[twoside,11pt]{article} 
\usepackage{amsmath,amsfonts,bm}
\usepackage{hyperref}
\usepackage{pgf,tikz}
\usetikzlibrary{arrows}

\usepackage{amsthm} 
\usepackage{amssymb}
\usepackage{framed,mdframed}
\usepackage{graphicx,color} 
\usepackage{mathrsfs,xcolor} 
\usepackage[all]{xy}
\usepackage{fancybox} 
% \usepackage{CJKutf8}
\usepackage{xeCJK}
\newtheorem{theorem}{定理}
\newtheorem{lemma}{引理}
\newtheorem{corollary}{推论}
\newtheorem*{exercise}{习题}
\newtheorem*{example}{例}
\setCJKmainfont[BoldFont=Adobe Heiti Std R]{Adobe Song Std L}
% \usepackage{latexdef}
\def\ZZ{\mathbb{Z}} \topmargin -0.40in \oddsidemargin 0.08in
\evensidemargin 0.08in \marginparwidth 0.00in \marginparsep 0.00in
\textwidth 16cm \textheight 24cm \newcommand{\D}{\displaystyle}
\newcommand{\ds}{\displaystyle} \renewcommand{\ni}{\noindent}
\newcommand{\pa}{\partial} \newcommand{\Om}{\Omega}
\newcommand{\om}{\omega} \newcommand{\sik}{\sum_{i=1}^k}
\newcommand{\vov}{\Vert\omega\Vert} \newcommand{\Umy}{U_{\mu_i,y^i}}
\newcommand{\lamns}{\lambda_n^{^{\scriptstyle\sigma}}}
\newcommand{\chiomn}{\chi_{_{\Omega_n}}}
\newcommand{\ullim}{\underline{\lim}} \newcommand{\bsy}{\boldsymbol}
\newcommand{\mvb}{\mathversion{bold}} \newcommand{\la}{\lambda}
\newcommand{\La}{\Lambda} \newcommand{\va}{\varepsilon}
\newcommand{\be}{\beta} \newcommand{\al}{\alpha}
\newcommand{\dis}{\displaystyle} \newcommand{\R}{{\mathbb R}}
\newcommand{\N}{{\mathbb N}} \newcommand{\cF}{{\mathcal F}}
\newcommand{\gB}{{\mathfrak B}} \newcommand{\eps}{\epsilon}
\renewcommand\refname{参考文献} \def \qed {\hfill \vrule height6pt
  width 6pt depth 0pt} \topmargin -0.40in \oddsidemargin 0.08in
\evensidemargin 0.08in \marginparwidth0.00in \marginparsep 0.00in
\textwidth 15.5cm \textheight 24cm \pagestyle{myheadings}
\markboth{\rm \centerline{}} {\rm \centerline{}}
\begin{document}
\title{\huge{\bf{利用模2约化法解决杭州师范大学2008年考研高代第1题}}}
\author{\small{叶卢庆\footnote{叶卢庆(1992---),男,杭州师范大学理学院数
      学与应用数学专业本科在读,E-mail:h5411167@gmail.com}}\\{\small{杭
      州师范大学理学院,浙江~杭州~310036}}} \date{}
\maketitle

% ----------------------------------------------------------------------------------------
% ABSTRACT AND KEYWORDS
% ----------------------------------------------------------------------------------------

\vspace{30pt} % Some vertical space between the abstract and first section
\begin{exercise}[杭州师范大学2008年考研高等代数]
  求证:多项式 $x^6+x^3+1$ 在有理数域上不可约.
\end{exercise}
\begin{proof}[\textbf{证明}]
  要证明 $f(x)=x^6+x^3+1$ 在 $\mathbf{Q}$ 上不可约,即只用证明
$$
f(x)=x^6+x^3+1
$$
在 $\mathbf{Z}$ 上不可约(因为在 $\mathbf{Q}$ 上可约的整系数多项式必定
在 $\mathbf{Z}$ 上可约).假若 $f(x)$在 $\mathbf{Z}$ 上可
约,则设 $f(x)=p(x)g(x)$,其中 $p(x),g(x)$ 都是整系数多项式.$p(x),g(x)$
中,必定有一个次数不大于3的非常数整系数因式,不妨设为 $p(x)$,此时易得 $f(x)$
是 $\mathbf{F}_2[x]$ 上的可约多项式.在模2下考
虑,$p(x)$ 只可能是如下几种多项式,所以在 $\mathbf{F}_2[x]$ 里, $f(x)$ 只可能被如下几种多项式整
除:
$$
x,x+1;x^2,x^2+x,x^2+1,x^2+x+1;x^3,x^3+1,x^3+x,x^3+x^{2},x^3+x^2+1,x^3+x^2+x,x^3+x+1,x^3+x^2+x+1
$$
但是经过 $\mathbf{F}_2[x]$ 中的带余除法验证,易得以上多项式都不整除 $f(x)$,因此假设错误.可得
$f(x)$ 在 $\mathbf{Z}$ 上不可约,因此在 $\mathbf{Q}$ 上也不可约.
\end{proof}
% ----------------------------------------------------------------------------------------
% ESSAY BODY
% ----------------------------------------------------------------------------------------

% BIBLIOGRAPHY
% ----------------------------------------------------------------------------------------
% 
% ----------------------------------------------------------------------------------------
\end{document}

我们选取四个
整数点 $-1,0,1,2$ 上作函数值 $f(-1)=1,f(0)=1,f(1)=3,f(2)=73$.易得
$p(-1)|f(-1)$,$p(0)|f(0)$,$p(1)|f(1)$,$p(2)|f(2)$.因此可得如下各种可能:
\begin{itemize}
\item $p(-1)=-1$,$p(0)=-1$,$p(1)=-1$,$p(2)=-1$.
\end{itemize}







经过点
$(-1,f(-1)),(0,f(0)),(1,f(1)),(2,f(2))$ 的 $3$ 次多项式是唯一存在的,用
手计算太麻烦,经过计算机,可得这样的3次多项式是
$$
11 x^3+x^2-10 x+1.
$$
经过点 $(-1,f(-1)),(0,f(0)),(1,f(1))$ 的二次多项式为
$$
x^2+x+1,
$$
经过点 $(-1,1),(0,1),(2,73)$ 的二次多项式为
$$
12x^2+12x+1,
$$
经过点 $(-1,1),(1,3),(2,73)$ 的二次多项式为
$$
23x^2+x-21,
$$
经过点 $(0,1),(1,3),(2,73)$ 的二次多项式为
$$
34x^2-32x+1.
$$
