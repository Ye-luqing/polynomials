\documentclass[twoside,11pt]{article} 
\usepackage{amsmath,amsfonts,bm}
\usepackage{hyperref}
\usepackage{amsthm} 
\usepackage{amssymb}
\usepackage{framed,mdframed}
\usepackage{graphicx,color} 
\usepackage{mathrsfs,xcolor} 
\usepackage[all]{xy}
\usepackage{fancybox} 
% \usepackage{CJKutf8}
\usepackage{xeCJK}
\newtheorem{theorem}{定理}
\newtheorem{lemma}{引理}
\newtheorem{corollary}{推论}
\newtheorem{question}{问题}
\newtheorem*{exercise}{习题}
\newtheorem*{example}{例}
\setCJKmainfont[BoldFont=Adobe Heiti Std R]{Adobe Song Std L}
% \usepackage{latexdef}
\def\ZZ{\mathbb{Z}} \topmargin -0.40in \oddsidemargin 0.08in
\evensidemargin 0.08in \marginparwidth 0.00in \marginparsep 0.00in
\textwidth 16cm \textheight 24cm \newcommand{\D}{\displaystyle}
\newcommand{\ds}{\displaystyle} \renewcommand{\ni}{\noindent}
\newcommand{\pa}{\partial} \newcommand{\Om}{\Omega}
\newcommand{\om}{\omega} \newcommand{\sik}{\sum_{i=1}^k}
\newcommand{\vov}{\Vert\omega\Vert} \newcommand{\Umy}{U_{\mu_i,y^i}}
\newcommand{\lamns}{\lambda_n^{^{\scriptstyle\sigma}}}
\newcommand{\chiomn}{\chi_{_{\Omega_n}}}
\newcommand{\ullim}{\underline{\lim}} \newcommand{\bsy}{\boldsymbol}
\newcommand{\mvb}{\mathversion{bold}} \newcommand{\la}{\lambda}
\newcommand{\La}{\Lambda} \newcommand{\va}{\varepsilon}
\newcommand{\be}{\beta} \newcommand{\al}{\alpha}
\newcommand{\dis}{\displaystyle} \newcommand{\R}{{\mathbb R}}
\newcommand{\N}{{\mathbb N}} \newcommand{\cF}{{\mathcal F}}
\newcommand{\gB}{{\mathfrak B}} \newcommand{\eps}{\epsilon}
\renewcommand\refname{参考文献} \def \qed {\hfill \vrule height6pt
  width 6pt depth 0pt} \topmargin -0.40in \oddsidemargin 0.08in
\evensidemargin 0.08in \marginparwidth0.00in \marginparsep 0.00in
\textwidth 15.5cm \textheight 24cm \pagestyle{myheadings}
\markboth{\rm \centerline{}} {\rm \centerline{}}
\begin{document}
\title{\huge{\bf{经过$n+1$个整点的$n$ 次多项式为本原多项式的充要条件}}}
\author{\small{叶卢庆\footnote{叶卢庆(1992---),男,杭州师范大学理学院数
      学与应用数学专业本科在读,E-mail:h5411167@gmail.com}}\\{\small{杭
      州师范大学理学院,浙江~杭州~310036}}} \date{}
\maketitle

% ----------------------------------------------------------------------------------------
% ABSTRACT AND KEYWORDS
% ----------------------------------------------------------------------------------------


\textbf{\small{摘要}:}探讨了经过 $n+1$ 个整点的$n$次多项式为本原多项式
的充要条件 \smallskip

\textbf{\small{关键词}:}Newton插值,本原多项式\smallskip


\vspace{30pt} % Some vertical space between the abstract and first section

% ----------------------------------------------------------------------------------------
% ESSAY BODY
% ----------------------------------------------------------------------------------------
\begin{question}
  设 $(x_1,y_1),\cdots,(x_n,y_{n}),(x_{n+1},y_{n+1})$ 为平面上
  的 $n+1$个整点.其中 $x_1,x_2,\cdots,x_{n+1}$ 两两不等.我们知道,经过
  这 $n+1$个整点的多项式只有一个.下面我们来探究经过这些点的多项式是本原
  多项式的充要条件.
\end{question}
\bigskip
在文章
\href{https://github.com/Ye-luqing/polynomials/blob/master/%E7%BB%8F%E8%BF%87%E5%B9%B3%E9%9D%A2%E4%B8%8An%2B1%E4%B8%AA%E6%95%B4%E7%82%B9%E7%9A%84%E5%A4%9A%E9%A1%B9%E5%BC%8F%E4%B8%BA%E6%95%B4%E7%B3%BB%E6%95%B0%E5%A4%9A%E9%A1%B9%E5%BC%8F%E7%9A%84%E5%85%85%E8%A6%81%E6%9D%A1%E4%BB%B6.tex}{
                                经过平面上$n+1$个整点的多项式为整系数多
                                项式的充要条件}里,我们已经表明,经过这
                              $n+1$ 个点的多项式 $f(x)$ 是整系数多项式
                              当且仅当
$$
f[x_1],f[x_1,x_2],\cdots,f[x_1,x_2,\cdots,x_{n+1}].
$$
里的每个数都是整数.其中$\forall k\geq 2$,
$$f[x_1,x_2,\cdots,x_k]=\frac{f[x_1,\cdots,x_{k-1}]-f[x_2,\cdots,x_k]}{x_1-x_k},$$
且 $f[x_1]=f(x_1)$,$f[x_2]=f(x_2)$.

  \begin{thebibliography}{}
    % \bibitem[1]{tao}Terence Tao.陶哲轩实分析[M].王昆扬,译.北京:人民邮
    %   电
    %   出版
    %   社,2008:376
  \end{thebibliography}
  % ----------------------------------------------------------------------------------------
\end{document}








